\documentclass[12pt]{article}
\usepackage[left=1cm, right=1cm, top=2cm,bottom=1.5cm]{geometry} 

\usepackage[parfill]{parskip}
\usepackage[utf8]{inputenc}
\usepackage[T2A]{fontenc}
\usepackage[russian]{babel}
\usepackage{enumitem}
\usepackage[normalem]{ulem}
\usepackage{amsfonts, amsmath, amsthm, amssymb, mathtools,xcolor}
\usepackage{blkarray}

\usepackage{tabularx}
\usepackage{hhline}

\usepackage{accents}
\usepackage{fancyhdr}
\pagestyle{fancy}
\renewcommand{\headrulewidth}{1.5pt}
\renewcommand{\footrulewidth}{1pt}

\usepackage{graphicx}
\usepackage[figurename=Рис.]{caption}
\usepackage{subcaption}
\usepackage{float}

%%Наименование папки откуда забирать изображения
\graphicspath{ {./images/} }

%%Изменение формата для ввода доказательства
\renewcommand{\proofname}{$\square$  \nopunct}
\renewcommand\qedsymbol{$\blacksquare$}

%%Изменение отступа на таблицах
\addto\captionsrussian{%
	\renewcommand{\proofname}{$\square$ \nopunct}%
}
%% Римские цифры
\newcommand{\RN}[1]{%
	\textup{\uppercase\expandafter{\romannumeral#1}}%
}

%% Для удобства записи
\newcommand{\MR}{\mathbb{R}}
\newcommand{\MC}{\mathbb{C}}
\newcommand{\MQ}{\mathbb{Q}}
\newcommand{\MN}{\mathbb{N}}
\newcommand{\MZ}{\mathbb{Z}}
\newcommand{\MTB}{\mathbb{T}}
\newcommand{\MTI}{\mathbb{I}}
\newcommand{\MI}{\mathrm{I}}
\newcommand{\MCI}{\mathcal{I}}
\newcommand{\MJ}{\mathrm{J}}
\newcommand{\MH}{\mathrm{H}}
\newcommand{\MT}{\mathrm{T}}
\newcommand{\MU}{\mathcal{U}}
\newcommand{\MV}{\mathcal{V}}
\newcommand{\MB}{\mathcal{B}}
\newcommand{\MF}{\mathcal{F}}
\newcommand{\MW}{\mathcal{W}}
\newcommand{\ML}{\mathcal{L}}
\newcommand{\MP}{\mathcal{P}}
\newcommand{\VN}{\varnothing}
\newcommand{\VE}{\varepsilon}
\newcommand{\dx}{\, dx}
\newcommand{\dy}{\, dy}
\newcommand{\dz}{\, dz}
\newcommand{\dd}{\, d}


\theoremstyle{definition}
\newtheorem{defn}{Опр:}
\newtheorem{rem}{Rm:}
\newtheorem{prop}{Утв.}
\newtheorem{exrc}{Упр.}
\newtheorem{problem}{Задача}
\newtheorem{lemma}{Лемма}
\newtheorem{theorem}{Теорема}
\newtheorem{corollary}{Следствие}

\newenvironment{cusdefn}[1]
{\renewcommand\thedefn{#1}\defn}
{\enddefn}

\DeclareRobustCommand{\divby}{%
	\mathrel{\text{\vbox{\baselineskip.65ex\lineskiplimit0pt\hbox{.}\hbox{.}\hbox{.}}}}%
}
\DeclareRobustCommand{\ndivby}{\mkern-1mu\not\mathrel{\mkern4.5mu\divby}\mkern1mu}


%Короткий минус
\DeclareMathSymbol{\SMN}{\mathbin}{AMSa}{"39}
%Длинная шапка
\newcommand{\overbar}[1]{\mkern 1.5mu\overline{\mkern-1.5mu#1\mkern-1.5mu}\mkern 1.5mu}
%Функция знака
\DeclareMathOperator{\sgn}{sgn}

%Функция ранга
\DeclareMathOperator{\rk}{\text{rk}}
\DeclareMathOperator{\diam}{\text{diam}}


%Обозначение константы
\DeclareMathOperator{\const}{\text{const}}

\DeclareMathOperator{\codim}{\text{codim}}

\DeclareMathOperator*{\dsum}{\displaystyle\sum}
\newcommand{\ddsum}[2]{\displaystyle\sum\limits_{#1}^{#2}}

%Интеграл в большом формате
\DeclareMathOperator{\dint}{\displaystyle\int}
\newcommand{\ddint}[2]{\displaystyle\int\limits_{#1}^{#2}}
\newcommand{\ssum}[1]{\displaystyle \sum\limits_{n=1}^{\infty}{#1}_n}

\newcommand{\smallerrel}[1]{\mathrel{\mathpalette\smallerrelaux{#1}}}
\newcommand{\smallerrelaux}[2]{\raisebox{.1ex}{\scalebox{.75}{$#1#2$}}}

\newcommand{\smallin}{\smallerrel{\in}}
\newcommand{\smallnotin}{\smallerrel{\notin}}

\newcommand*{\medcap}{\mathbin{\scalebox{1.25}{\ensuremath{\cap}}}}%
\newcommand*{\medcup}{\mathbin{\scalebox{1.25}{\ensuremath{\cup}}}}%

\makeatletter
\newcommand{\vast}{\bBigg@{3.5}}
\newcommand{\Vast}{\bBigg@{5}}
\makeatother

%Промежуточное значение для sup\inf, поскольку они имеют разную высоту
\newcommand{\newsup}{\mathop{\smash{\mathrm{sup}}}}
\newcommand{\newinf}{\mathop{\mathrm{inf}\vphantom{\mathrm{sup}}}}

%Скалярное произведение
\newcommand{\inner}[2]{\left\langle #1, #2 \right\rangle }
\newcommand{\linsp}[1]{\left\langle #1 \right\rangle }
\newcommand{\linmer}[2]{\left\langle #1 \vert #2\right\rangle }

%Подпись символов снизу
\newcommand{\ubar}[1]{\underaccent{\bar}{#1}}

%% Шапка для букв сверху
\newcommand{\wte}[1]{\widetilde{#1}}
\newcommand{\wht}[1]{\widehat{#1}}
\newcommand{\ovl}[1]{\overline{#1}}

%%Трансформация Фурье
\newcommand{\fourt}[1]{\mathcal{F}\left(#1\right)}
\newcommand{\ifourt}[1]{\mathcal{F}^{-1}\left(#1\right)}

%%Символ вектора
\newcommand{\vecm}[1]{\overrightarrow{#1\,}}

%%Пространстов матриц
\newcommand{\matsq}[1]{\operatorname{Mat}_{#1}}
\newcommand{\mat}[2]{\operatorname{Mat}_{#1, #2}}

%Оператор для действ и мнимых чисел
\DeclareMathOperator{\IM}{\operatorname{Im}}
\DeclareMathOperator{\RE}{\operatorname{Re}}
\DeclareMathOperator{\li}{\operatorname{li}}
\DeclareMathOperator{\GL}{\operatorname{GL}}
\DeclareMathOperator{\SL}{\operatorname{SL}}
\DeclareMathOperator{\Char}{\operatorname{char}}
\DeclareMathOperator\Arg{Arg}

%Делимость чисел
\newcommand{\modn}[3]{#1 \equiv #2 \; (\bmod \; #3)}


%%Взятие в скобки, модули и норму
\newcommand{\parfit}[1]{\left( #1 \right)}
\newcommand{\modfit}[1]{\left| #1 \right|}
\newcommand{\sqparfit}[1]{\left\{ #1 \right\}}
\newcommand{\normfit}[1]{\left\| #1 \right\|}

%%Функция для обозначения равномерной сходимости по множеству
\newcommand{\uconv}[1]{\overset{#1}{\rightrightarrows}}
\newcommand{\uconvm}[2]{\overset{#1}{\underset{#2}{\rightrightarrows}}}


%%Функция для обозначения нижнего и верхнего интегралов
\def\upint{\mathchoice%
	{\mkern13mu\overline{\vphantom{\intop}\mkern7mu}\mkern-20mu}%
	{\mkern7mu\overline{\vphantom{\intop}\mkern7mu}\mkern-14mu}%
	{\mkern7mu\overline{\vphantom{\intop}\mkern7mu}\mkern-14mu}%
	{\mkern7mu\overline{\vphantom{\intop}\mkern7mu}\mkern-14mu}%
	\int}
\def\lowint{\mkern3mu\underline{\vphantom{\intop}\mkern7mu}\mkern-10mu\int}

%%След матрицы
\DeclareMathOperator*{\tr}{tr}

\makeatletter
\renewcommand*\env@matrix[1][*\c@MaxMatrixCols c]{%
	\hskip -\arraycolsep
	\let\@ifnextchar\new@ifnextchar
	\array{#1}}
\makeatother


%% Переопределение функции хи, чтобы выглядела более приятно
\makeatletter
\@ifdefinable\@latex@chi{\let\@latex@chi\chi}
\renewcommand*\chi{{\@latex@chi\smash[t]{\mathstrut}}} % want only bottom half of \mathstrut
\makeatletter

\setcounter{MaxMatrixCols}{20}

\begin{document}
\lhead{Функциональный анализ-\RN{1}}
\chead{Шейпак И.А.}
\rhead{Лекция - 1}

\section*{Введение}
\textbf{План $1$-го полугодия}
\begin{enumerate}[label=\arabic*)]
	\item Метрические и нормированные пространства;
	\item Банаховы пространства и геометрия в них;
	\item Гильбертовы пространства и геометрия в них;
	\item Линейные непрерывные функционалы и операторы в банаховых и гильбертовых пространствах;
	\item Спектральная теория;
	\item Теория Фредгольма;
\end{enumerate}

\section*{Метрические и нормированные пространства}
\subsection*{Метрические пространства}
\begin{defn}
	Пару $(X,\rho)$ будем называть \uwave{метрическим пространством}, где $X$ - произвольное множество, функция расстояния $\rho$ задается следующим образом:
	$$
		\rho \colon X\times X \to \MR^{+}_0 = [0; +\infty)
	$$
	называется \uwave{метрикой} и обладает свойствами, называемыми \uwave{аксиомами метрики}:
	\begin{enumerate}[label=\arabic*)]
		\item \textbf{\uline{Неотрицательность}}: $\forall x,y \in X, \, \rho(x,y) \geq 0 \colon \rho(x,y) = 0 \Leftrightarrow x = y$;
		\item \textbf{\uline{Симметричность}}: $\forall x,y \in X, \, \rho(x,y) = \rho(y,x)$;
		\item \textbf{\uline{Неравенство треугольника}}: $\forall x,y,z \in X, \, \rho(x,y) \leq \rho(x,z) + \rho(z,y)$;
	\end{enumerate}
\end{defn}

\textbf{Примеры метрических пространств}:
\begin{enumerate}[label=\arabic*)]
	\item $x = (x_1,\dotsc,x_n) \in \MR^n, \, \forall i, \, x_i\in \MR$ с метрикой:
	$$
		\rho(x,y) = \sqrt{\ddsum{i = 1}{n}|x_i - y_i|^2}
	$$ 
	\item $C[a,b]$ - непрерывные на $[a,b]$ функции с метрикой:
	$$
		\rho(f,g) = \max\limits_{x \in [a,b]}|f(x) - g(x)|
	$$
	\item \uwave{Дискретное пространство}: $X$ - произвольное непустое множество с метрикой:
	$$
		\rho(x,y) = 
		\begin{cases}
			0, & x = y \\
			1, & x \neq y
		\end{cases}
	$$
\end{enumerate}
\newpage
\subsection*{Сходимость в $(X,\rho)$}
\begin{defn}
	Последовательность $\{x_n\}_{n = 1}^{\infty}\subset X$ \uwave{сходится} к $x$: $x_n \to x$, если $\rho(x_n,x) \xrightarrow[n \to \infty]{} 0$.
\end{defn}
\begin{defn}
	Последовательность $\{x_n\}_{n = 1}^{\infty}\subset X$ называется \uwave{фундаментальной}, если:
	$$
		\forall \VE > 0, \, \exists \, N = N(\VE) \colon \forall n,m\geq N, \, \rho(x_n, x_m) < \VE
	$$
\end{defn}
\begin{defn}
	Метрическое пространство $(X,\rho)$ называется \uwave{полным}, если $\forall \{x_n\}_{n = 1}^{\infty}\subset X$ - фундаментальная:
	$$
		\exists \lim\limits_{n\to\infty}x_n = x \in X
	$$
\end{defn}
Полные пространства хороши тем, что если мы делаем предельный переход в нем, то нам не нужно заботиться о том, что предельный элемент обладает теми же свойствами, что и элементы $x_n$. 

Например, пространство непрерывных функций - полное $\Rightarrow$ если делаем предельный переход по метрике на этом пространстве (равномерный предел непрерывных функций это непрерывная функция), то получим непрерывную функцию.

Можно рассматривать пространство интегрируемых функций, они тоже полны $\Rightarrow$ предельный переход даст интегрируемую функцию.

\textbf{Примеры метрических пространств}:
\begin{enumerate}[label=\arabic*)]
	\item $(\MR^n,\rho)$ - полное;
	\item $(C[a,b],\rho)$ - полное;
	\item Дискретное пространство - полное;
	\item $\MQ$ - пространство рациональных чисел с метрикой $\rho(x,y) = |x - y|$ - не является полным. Можно взять: $\sqrt{2} \not\in \MQ$, а в качестве $x_n$ возьмем последовательность цифр, которая приближается к $\sqrt{2}$:
	$$
		x_n = 1,\underbrace{4\dotsc\dotsc}_{n} \xrightarrow[n \to \infty]{}x = x\sqrt{2}
	$$
	Эта последовательность фундаментальна в $\MR$, но её предел $\not\in \MQ$;
	\item Пространство \uwave{финитных последовательностей}: $c_{00}\colon x = (x_1,x_2, \dotsc, x_n, 0, 0 , \dotsc) \in c_{00}$:
	$$
		\forall x \in c_{00}, \, \exists \, N = n(x) \colon \forall k > n, \, x_k = 0
	$$
	Это пространство не полно, рассмотрим последовательность:
	$$
		x^k = \left(1,\tfrac{1}{2}, \dotsc, \tfrac{1}{k}, 0 , \dotsc \right) \in c_{00}
	$$
	Она фундаментальна и после $k$-го номера первые $k$ координат заведомо не меняются $\Rightarrow$ предельная последовательность не будет финитной (все координаты будут ненулевыми);
\end{enumerate}
\newpage
\subsection*{Геометрические понятия в метрических пространствах}
\begin{defn}
	\uwave{Открытым шаром} $\MB(x_0,r)$ в метрическом пространстве $(X,\rho)$ с центром в точке $x_0$ радиуса $r$ называется множество:
	$$
		\MB(x_0,r) \coloneqq\{x \in X \colon \rho(x,x_0) < r\}
	$$
\end{defn}
\begin{defn}
	\uwave{Замкнутым шаром} $\ovl{\MB(x_0,r)} = \MB[x_0,r]$ в метрическом пространстве $(X,\rho)$ с центром в точке $x_0$ радиуса $r$ называется множество:
	$$
		\MB[x_0,r] \coloneqq\{x \in X \colon \rho(x,x_0) \leq r\}
	$$
\end{defn}
Пусть $M$ - множество в метрическом пространстве $(X,\rho)$: $M\subset (X,\rho)$.
\begin{defn}
	Точка $x_0$ является \uwave{предельной точкой} множества $M$, если: 
	$$
		\forall \VE > 0, \, \MB(x_0,\VE) \cap M
	$$ 
	содержит бесконечно много точек.
\end{defn}
\begin{corollary}
	Точка $x_0$ - предельная точка для $M \Leftrightarrow \exists \, \{x_n\}_{n = 1}^{\infty}, \, x_n \in M$ и $x_n \xrightarrow{\rho}x_0$.
\end{corollary}
\begin{proof}\hfill\\
	$(\Leftarrow)$ $\forall \VE > 0, \, \exists \, N = N(\VE) \in \MN \colon \forall n \geq N, \, \rho(x_n,x_0) < \VE \Rightarrow x_n \in \MB(x_0,\VE)$, видно что этих элементов бесконечное число в силу $\forall n \geq N$.
	
	$(\Rightarrow)$ $\forall n \in \MN, \, \MB\left(x_0,\tfrac{1}{n}\right)\cap M$ содержит бесконечно много точек, тогда:
	$$
		\forall n, \, \exists \, x_n \in \MB\left(x_0, \tfrac{1}{n}\right)\cap M \Rightarrow x_n \in M
	$$ 
	Поскольку их бесконечно много, то мы можем выбирать их разными, то есть, чтобы $x_n \neq x_1,x_2, \dotsc, x_{n-1}$. В итоге, мы построили последовательность: $x_n \to x_0$, где $\forall n, \, x_n \in M$. 
\end{proof}

\begin{defn}
	Точка $x_0$ называется \uwave{точкой прикосновения} множества $M \subset (X,\rho)$, если:
	$$
		\forall \VE > 0, \, \MB(x_0,\VE) \cap M \neq \VN
	$$
\end{defn}
\begin{defn}
	Точка $x_0$ называется \uwave{изолированной точкой} множества $M \subset (X,\rho)$, если:
	$$
		\exists\, \VE > 0 \colon \MB(x_0,\VE) \cap M =\{x_0\}
	$$
\end{defn}
\begin{corollary}
	Если $x_0$ - предельная точка, то $x_0$ - это точка прикосновения. Обратно, вообще говоря, не верно.
\end{corollary}
\begin{proof}
	Если $x_0$ - предельная точка, то по определению $\forall \VE > 0, \, \MB(x_0,\VE) \cap M \neq \VN$, поскольку содержит бесконечно много точек $\Rightarrow$ точка прикосновения.
	
	Обратно, может быть не верно, когда $\exists\, \VE > 0, \,\MB(x_0,\VE) \cap M \neq \VN$, но при этом такое пересечение содержит конечное число точек, то есть $x_0$ - точка прикосновения, но не предельная:
	$$
		\exists \, \VE > 0,\, \MB(x_0,\VE) \cap M \neq \VN, \, \MB(x_0,\VE) \cap M = \{x_1,x_2,\dotsc,x_m\}
	$$
	Более того, если мы возьмем: $0 < \hat{\VE} \leq \min\limits_{1 \leq i \leq m}\rho(x_0,x_i)$, то $\MB(x_0,\VE) \cap M = \{x_0\}$ - \uwave{изолированная точка}.
\end{proof}

\begin{defn}
	\uwave{Замыканием множества} $M \subset (X,\rho)$ называется объединение множества $M$ со всеми его предельными точками:
	$$
		\ovl{M} \coloneqq M\cup \{\text{все предельные точки } M\}
	$$
\end{defn}
\begin{exrc}
	Для любого множества $M\subset (X,\rho)$, его замыкание $\ovl{M} \equiv$ все точки прикосновения $M$. 
\end{exrc}
\begin{proof}\hfill\\
	$(\Rightarrow)$ Пусть $x \in \ovl{M}$ - предельная точка $M$, тогда по следствию выше $x$ - точка прикосновения $M$. Если $x \in M$ - не является предельной, то $\exists \, \VE > 0 \colon \MB(x,\VE) \cap M$ - содержит конечное число точек $\Rightarrow x$ это изолированная точка $\Rightarrow$ точка прикосновения $M$.
	
	$(\Leftarrow)$ Пусть $x \in \ovl{M}$ - точка прикосновения $M \Rightarrow \forall \VE > 0, \, \MB(x,\VE)\cap M \neq\VN$. Пусть $x \in M$ тогда:
	$$
		x \in M \Rightarrow x \in \MB(x,\VE) \Rightarrow x \in \MB(x,\VE)\cap M \neq \VN
	$$
	Пусть $x \not\in M$, тогда $x$ может быть только предельной точкой. Пусть это не так, тогда:
	$$
		\exists \, \VE > 0 \colon \MB(x,\VE) \cap M = \{x_1,x_2,\dotsc,x_m\} \Rightarrow \exists \, 0 < \hat{\VE} \leq \min\limits_{1 \leq i \leq m}\rho(x_0,x_i) \colon \MB(x,\hat{\VE}) \cap M = \VN
	$$
	Получили противоречие, поскольку $ \forall \VE > 0, \, \MB(x,\VE)\cap M \neq\VN$.
\end{proof}
\begin{defn}
	Множество $M \subset (X,\rho)$ называется \uwave{открытым}, если: 
	$$
		\forall x \in M, \, \exists\, \VE > 0 \colon \MB(x,\VE) \subset M
	$$ 
	то есть каждая точка содержится вместе с некоторым шаром.
\end{defn}
\begin{defn}
	Множество $M \subset (X,\rho)$ называется \uwave{замкнутым}, если его дополнение $X \setminus M$ - открыто.
\end{defn}

\begin{exrc}\textbf{\uline{Свойства открытых множеств}}:
	\begin{enumerate}[label=\arabic*)]
		\item $\MB(x_0, r)$ - открытое множество;
		\item $\MB[x_0,r]$ - замкнутое множество;
		\item $\ovl{M}$ - замкнутое множество;
		\item $\ovl{\ovl{M}} = \ovl{M}$ - замкнутое множество содержит все свои предельные точки;
	\end{enumerate}
\end{exrc}
\begin{proof}\hfill
	\begin{enumerate}[label=\arabic*)]
		\item Уже доказывали в математическом анализе. По определению:
		$$
			\MB(x_0,r) = \{x \in X \colon \rho(x,x_0) < r\} \Rightarrow \forall x \in \MB(x_0,r), \, \exists \, \VE = r - \rho(x,x_0) \Rightarrow \MB(x,\VE) \subset \MB(x_0,r)
		$$
		$$
			\forall y \in \MB(x,\VE), \, \rho(x_0,y) \leq \rho(x_0, x) + \rho(x,y) < \rho(x_0,x) + \VE = \rho(x_0,x) + r - \rho(x_0,x) = r
		$$
		\item Пусть $x \in X\setminus\MB[x_0,r]$, тогда по определению $\rho(x,x_0) > r$. Пусть $\VE = \rho(x,x_0) - r > 0$, тогда:
		$$
			\forall y \in \MB(x,\VE), \, \rho(x_0,x) \leq \rho(x_0,y) + \rho(y,x) < \rho(x_0,y) + \VE = \rho(x_0,y) + \rho(x_0,x) - r \Rightarrow
		$$
		$$
			\Rightarrow \rho(x_0, y) > r \Rightarrow y \in X \setminus \MB[x_0,r] \Rightarrow \MB(x,\VE)\subset X\setminus \MB[x_0,r]
		$$
		\item По определению, $\ovl{M}$ содержит $M$ и все предельные точки $M$, тогда: 
		$$
			\forall x \in X\setminus \ovl{M}, \, \exists \, \VE > 0 \colon \MB(x,\VE) \cap M = \VN  
		$$ 
		Покажем, что из этого следует: $\MB(x,\VE) \cap \ovl{M} = \VN$. Пусть $\exists \, y \in \MB(x,\VE) \colon y \in \MB(x,\VE) \cap \ovl{M}$. Поскольку $y \not\in M$, то $y$ - предельная точка $M$. Пусть $\VE_1 = \VE - \rho(x,y)$, тогда:
		$$
			\forall z \in \MB(y,\VE_1), \, \rho(z,x) \leq \rho(z,y) + \rho(y,x) < \VE_1 + (\VE - \VE_1) = \VE \Rightarrow \MB(y,\VE_1) \subset \MB(x,\VE)
		$$
		Но $\MB(x,\VE) \cap M = \VN \Rightarrow y$ не может быть предельной точкой $M \Rightarrow$ противоречие. Тогда:
		$$
			\exists \, \VE > 0 \colon \MB(x,\VE) \cap M = \VN  \Rightarrow \MB(x,\VE) \cap \ovl{M} = \VN \Rightarrow \MB(x,\VE) \subset X \setminus \ovl{M}
		$$
		\item В одну сторону это очевидно, поскольку $x \in \ovl{M} \Rightarrow x \in \ovl{\ovl{M}}$. Пусть $x \in \ovl{\ovl{M}}$, тогда:
		$$
			\forall \VE > 0, \, \MB(x,\VE) \cap \ovl{M} \neq \VN \Rightarrow \exists \, y \in \MB(x,\VE) \cap \ovl{M}, \, y \in \ovl{M} 
		$$
		Рассмотрим шар вокруг точки $y \in \ovl{M}$, пусть $\VE - \rho(x,y) = \VE_1$, тогда:
		$$
			 \forall z \in \MB(y,\VE_1), \, \rho(z,y) < \VE_1 \Rightarrow \rho(z,x) \leq \rho(z,y) + \rho(y,x) < \VE_1 + (\VE - \VE_1) = \VE \Rightarrow \MB(y,\VE_1) \subset \MB(x,\VE)
		$$
		$$
			y \in \ovl{M} \Rightarrow \exists \, y' \in \MB(y,\VE_1) \colon y' \in M \Rightarrow y' \in \MB(x,\VE)
		$$ 
		В силу произвольности $\MB(x,\VE)$ верно, что $x \in \ovl{M}$.
	\end{enumerate}
\end{proof}

\subsection*{Свойства полных метрических пространств}
\end{document}